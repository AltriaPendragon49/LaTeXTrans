\section{结论}
\label{sec:conclusions}

我们推导了神经感受野中涌现局部化的最小示例的有效学习动态,如 \textcite{ingrosso2022data} 所述。
我们采用的分析方法依赖于以下假设:\emph{条件}前激活是高斯分布的,这是一种对之前工作假设的改进,之前的工作假设无条件前激活是高斯分布的,如 \textcite{ingrosso2022data} 所提到的\emph{高斯等效性属性}。
这一方法可能能够扩展到我们专门的设置之外,并且可能为进一步分析输入任务的统计特性如何推动神经网络学习中的涌现结构提供了可能。

作为与稀疏性等自上而下约束的替代机制,涌现机制与近期的工作一致,后者重新将数据分布特性作为复杂行为的驱动力~\parencite{chan2022data}。
通过这些分析有效动态,我们观察到特定的数据属性——协方差结构和边际——在神经感受野中的局部化起到了塑造作用。
尽管我们无法通过单神经元分析模型捕捉神经元之间可能影响感受野的动态相互作用,但我们对多个神经元的实证验证表明,这些相互作用实际上在局部化的形成中并不起重要作用~\cite[\cf][]{harsh2020placecell}。

我们考虑的数据模型是对早期感觉系统面临的任务的简化抽象,因此,我们尚未捕捉到在早期感觉系统中观察到的感受野的某些特征。
特别是,我们没有观察到定向或相位选择性,这是早期感觉皮层和人工神经网络中简单细胞感受野的特征,能够在 \cref{fig:sim-real-gabors} 中某些感受野的子集中看到(分别位于左侧和中间)。
为了捕捉定向选择性,跟随 \textcite{karklin2011efficient} 的方法可能会有所帮助,后者将基于群体的高效编码框架中的定向选择性与噪声的存在联系起来。
此外,基于中心-周围过滤的输入数据,包括理想化数据,在我们的仿真中给出了具有子区域的感受野,但这一点难以分析。
最后,我们尚未研究感受野形状的分布,也没有与其他感受野学习模型进行验证,除了与 ICA 的简短比较~\parencite[\cf][]{saxe2011unsupervised},但这些是未来工作的令人兴奋的方向。
