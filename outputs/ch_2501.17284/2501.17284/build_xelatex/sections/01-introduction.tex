\section{引言}
\label{sec:introduction}

动物神经系统中外周反应的一个显著特征是 \emph{局部化}——也就是说,简单细胞神经元的线性感受野通常只对远小于其整个输入域的连续区域作出反应。
在视觉系统中,视网膜神经节细胞近似于对中心-周边区域的局部滤波器,这些滤波器铺满输入空间~\parencite{dacey2000center,doi2012efficient,knudsen1978space};
而在下游的初级视觉皮层中,简单细胞具有对空间频率和方向选择性的局部滤波器~\parencite{hubel1959receptive,hubel1968receptive,rolls1995sparseness,niell2008highly,willmore2011sparse,ringach2002orientation,ringach2002spatial}。
在初级躯体感觉皮层中,神经元对皮肤上受限区域的刺激产生反应~\parencite{crochet2011synaptic};
在初级听觉皮层中,时空感受野通常在时间和频率域中都是局部化的~\parencite{deweese2003binary,hromadka2008sparse};
参见 \cref{fig:sim-real-gabors}(左)。

相比之下,人工学习系统并不总是学习出局部化的滤波器。
主成分分析倾向于拟合跨越整个输入信号的权重,未加正则化的自编码神经网络架构和受限玻尔兹曼机也有类似表现~\parencite{saxe2011unsupervised}。
这一差异促使人们寻求能够从自然刺激中学习出局部感受野的人工学习模型,其中最具代表性的是稀疏编码~\parencite{olshausen1996emergence,olshausen1997sparse} 和独立成分分析~\parencite[ICA;~][]{bell1997independent,vanhateren1998independent}。
稀疏编码、ICA 及其他从自然数据中产生局部感受野的压缩方法采用的是自上而下的方式——它们通过优化显式的稀疏性准则,或在关键参数条件下需要稀疏性的独立性准则,从而找到输入信号的高效表示~\cite{field1999wavelets,saxe2011unsupervised}。

尽管稀疏性作为局部化的潜在统一解释具有吸引力,但局部化也会自然地出现在训练用于分类任务的网络中,即使这些网络并未显式使用稀疏性正则项~\parencite{krizhevsky2012imagenet,zeiler2013visualizing,yosinski2015understanding,sengupta2018manifoldtiling};
参见 \cref{fig:sim-real-gabors}(中)中的一个示例。
\textcite{ingrosso2022data} 提炼了此类涌现性局部化的示例,并展示了在简单前馈神经网络中,训练数据模型具有近似自然视觉输入特性的前提下,局部感受野会自然涌现出来,具体包括:
局部结构(非共址维度的统计独立性)以及非高斯性(高阶累积量非零)。
在模拟中,\textcite{ingrosso2022data} 将局部化的动态涌现与输入的高阶统计量调谐增强联系起来,并表明在这种设置下,甚至单个神经元也足以学习出局部感受野。

在本研究中,我们在 \textcite{ingrosso2022data} 的成果基础上进一步探究,旨在描述在该最小设置中局部感受野涌现背后的机制。
驱动局部化的高阶输入统计特性难以用现有依赖高斯性假设的分析工具进行研究~\parencite{goldt2020modelling}。
通过将学习过程划分为两个阶段,我们能够推导出在理想化自然数据驱动下,单个神经元模型在早期时间段的有效学习动力学方程。
我们的解析模型对驱动涌现的高阶统计特性给出了简洁的描述,
并通过多神经元模拟验证了该模型的正面与负面预测;参见 \cref{fig:sim-real-gabors}(右)。
这些发现提出了一种解释早期神经响应中普遍存在的局部化现象的替代路径,即源于神经回路中学习的非线性动态与具有高阶统计结构的自然数据之间的相互作用,
而非源于显式的效率准则。
