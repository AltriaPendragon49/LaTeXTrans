局部感受野——对输入的某些连续时空特征具有选择性的神经元——存在于哺乳动物大脑的早期感觉区域。优化显式稀疏性或独立性标准的无监督学习算法复制了这些局部感受野的特征,但未能直接解释在没有高效编码的情况下,如何通过学习实现局部化,正如在深度神经网络的早期层中发生的那样,也可能在生物系统的早期感觉区域中发生。我们考虑了一种替代模型,在该模型中,局部感受野在没有显式自上而下效率约束的情况下出现——一个基于自然图像结构启发的数据模型训练的前馈神经网络。先前的研究确定了非高斯统计对局部化的重要性,但对于推动动态涌现的机制仍然存在未解之谜。我们通过推导单个非线性神经元的有效学习动力学来解决这些问题,精确说明输入数据的高阶统计特性如何驱动涌现的局部化,并证明这些有效动力学的预测可以扩展到多神经元设置。我们的分析为局部化的普遍性提供了一种替代解释,认为局部化是神经电路中学习的非线性动力学的结果。\smash{\footnotemark}\footnotetext{
实验和图形的复现代码请见
\url{https://github.com/leonlufkin/localization}.
}
