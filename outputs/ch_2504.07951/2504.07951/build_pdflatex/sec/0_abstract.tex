\vspace{-0.5cm}
\begin{abstract}

构建能够通过多模态信号有效感知世界的通用模型一直是一个长期的目标。目前的方法包括集成独立预训练的组件,如将视觉编码器与LLM连接并\edit{继续进行多模态训练}。尽管这些方法展示了显著的样本效率,但是否这种后融合架构天生优越仍是一个未解的问题。在本研究中,我们重新审视了本地多模态模型(NMMs)的架构设计——即从零开始在所有模态上进行训练的模型,并进行了广泛的扩展定律研究,涵盖了457个具有不同架构和训练组合的训练模型。我们的研究表明,后融合架构并没有比不依赖图像编码器的早期融合架构具有固有的优势。相反,早期融合在较低参数数量下展现出更强的性能,更高效的训练,并且更容易部署。受到早期融合架构强大性能的启发,我们展示了结合专家混合(MoEs)能够让\edit{模型学习模态特定的权重,从而显著提升性能。}

\end{abstract}
