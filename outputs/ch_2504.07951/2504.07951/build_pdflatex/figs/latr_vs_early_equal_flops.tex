\begin{figure*}[t!]
    \centering
    \captionsetup{type=figure}
    \begin{subfigure}[t]{0.33\linewidth}
        \begin{tikzpicture}
    \begin{axis}[
        legend pos=north east,
        grid=major, %
        grid style={line width=.1pt, draw=gray!30}, %
        major grid style={line width=.2pt,draw=gray!50},
        minor tick num=2,
        axis x line*=bottom,
        axis y line*=left,
        xmode=log, %
        log basis x=10, %
        height=1.7in,
        width=1.05\linewidth,
        ylabel style={align=center, font=\scriptsize, yshift=-1ex},
        xlabel style={font=\scriptsize},
        ylabel={\footnotesize{Validation Loss}},
        xlabel={\scriptsize{FLOPs}},
        title=\footnotesize{Image-Caption},
        yticklabel style={font=\scriptsize, /pgf/number format/fixed, /pgf/number format/precision=1},
        xticklabel style={font=\scriptsize},
        mark options={solid},
        legend style={cells={align=left}, font=\footnotesize, text=black}, %
        legend columns=6, %
        legend cell align={left},
        legend to name=sharedlegend,
    ]
    \addplot[legend late_0_2b style] plot coordinates {
        (8.1e+19, 2.967)
        (1.62e+20, 2.874)
        (3.24e+20, 2.797)
        (6.48e+20, 2.728)

    };

    \addplot[legend late_0_4b style] plot coordinates {
        (1.37e+20, 2.869)
        (2.74e+20, 2.771)
        (5.48e+20, 2.695)
        (1.1e+21, 2.626)

    };

    \addplot[legend late_0_9b style] plot coordinates {
        (2.78e+20, 2.734)
        (5.56e+20, 2.629)
        (1.11e+21, 2.541)
        (2.22e+21, 2.467)

    };



    \addplot[legend late_2_2b style] plot coordinates {
        (1.34e+21, 2.466)
        (2.69e+21, 2.367)
        (5.37e+21, 2.287)
        (8.06e+21, 2.255)
    };

        







    \addplot[legend early_0_2b style] plot coordinates {
        (8.25e+19, 2.95)
        (1.65e+20, 2.865)
        (3.3e+20, 2.781)
        (6.6e+20, 2.715)


    };
    \addplot[legend early_0_4b style] plot coordinates {
        (1.39e+20, 2.852)
        (2.78e+20, 2.755)
        (5.57e+20, 2.663)
        (1.11e+21, 2.596)


    };
    \addplot[legend early_0_9b style] plot coordinates {
        (2.8e+20, 2.716)
        (5.59e+20, 2.598)
        (1.12e+21, 2.495)
        (2.24e+21, 2.425)


    };

        
    \addplot[legend early_2_2b style] plot coordinates {
        (1.37e+21, 2.455)
        (2.74e+21, 2.352)
        (5.47e+21, 2.279)
        (8.21e+21, 2.242)
    };
        
    

    \end{axis}
\end{tikzpicture}
    \end{subfigure}
    \begin{subfigure}[t]{0.32\linewidth}
        \begin{tikzpicture}
    \begin{axis}[
        legend pos=north east,
        grid=major,
        grid style={line width=.1pt, draw=gray!30},
        major grid style={line width=.2pt,draw=gray!50},
        minor tick num=2,
        axis x line*=bottom,
        axis y line*=left,
        xmode=log,
        log basis x=10,
        height=1.7in,
        width=1.05\linewidth,
        ylabel style={align=center, font=\scriptsize, yshift=-1ex},
        xlabel style={font=\scriptsize},
        xlabel={\scriptsize{FLOPs}},
        title=\footnotesize{Interleaved},
        yticklabel style={font=\scriptsize, /pgf/number format/fixed, /pgf/number format/precision=1},
        xticklabel style={font=\scriptsize},
        mark options={solid},
        legend style={cells={align=left}, font=\footnotesize, text=black},
        legend columns=6,
        legend cell align={left},
        legend to name=sharedlegend,
    ]

\addplot[legend late_0_2b style] plot coordinates {
        (8.1e+19, 3.113)
        (1.62e+20, 3.044)
        (3.24e+20, 2.988)
        (4.86e+20, 2.959)
        (6.48e+20, 2.94)
    };

    \addplot[legend late_0_4b style] plot coordinates {
        (1.37e+20, 3.013)
        (2.74e+20, 2.942)
        (5.48e+20, 2.883)
        (8.22e+20, 2.853)
        (1.1e+21, 2.833)

    };

    \addplot[legend late_0_9b style] plot coordinates {
        (2.78e+20, 2.899)
        (5.56e+20, 2.811)
        (1.11e+21, 2.741)
        (2.22e+21, 2.685)

    };
    \addlegendentry{Late-1B}

\addplot[legend late_2_2b style] plot coordinates {
        (1.34e+21, 2.697)
        (2.69e+21, 2.621)
        (4.03e+21, 2.58)
        (5.37e+21, 2.557)
        (8.06e+21, 2.527)

    };
    \addlegendentry{Late-2.4B}

\addplot[legend early_0_2b style] plot coordinates {
        (8.25e+19, 3.094)
        (1.65e+20, 3.022)
        (3.3e+20, 2.967)
        (4.95e+20, 2.939)
        (6.6e+20, 2.921)
    };

    \addplot[legend early_0_4b style] plot coordinates {
        (1.39e+20, 2.998)
        (2.78e+20, 2.927)
        (5.57e+20, 2.866)
        (8.35e+20, 2.836)
        (1.11e+21, 2.817)
    };

    \addplot[legend early_0_9b style] plot coordinates {
        (2.8e+20, 2.885)
        (5.59e+20, 2.795)
        (1.12e+21, 2.726)
        (2.24e+21, 2.669)
    };

    \addlegendentry{Early-0.9B}

\addplot[legend early_2_2b style] plot coordinates {
        (1.37e+21, 2.685)
        (2.74e+21, 2.61)
        (4.1e+21, 2.572)
        (5.47e+21, 2.542)
        (8.21e+21, 2.514)

    };
    \addlegendentry{Early-2.2B}

\end{axis}
\end{tikzpicture}

    \end{subfigure}
    \begin{subfigure}[t]{0.32\linewidth}
    \begin{tikzpicture}
    \begin{axis}[
        legend pos=north east,
        grid=major,
        grid style={line width=.1pt, draw=gray!30},
        major grid style={line width=.2pt,draw=gray!50},
        minor tick num=2,
        axis x line*=bottom,
        axis y line*=left,
        xmode=log,
        log basis x=10,
        height=1.7in,
        width=1.05\linewidth,
        ylabel style={align=center, font=\scriptsize, yshift=-1ex},
        xlabel style={font=\scriptsize},
        xlabel={\scriptsize{FLOPs}},
        title=\footnotesize{Text-only},
        yticklabel style={font=\scriptsize, /pgf/number format/fixed, /pgf/number format/precision=1},
        xticklabel style={font=\scriptsize},
        mark options={solid},
        legend style={cells={align=left}, font=\scriptsize, text=black},
        legend columns=4,
        legend cell align={left},
        legend to name=sharedlegend,
    ]

    \addplot[legend late_0_2b style] plot coordinates {
        (8.1e+19, 3.349)
        (1.62e+20, 3.281)
        (3.24e+20, 3.226)
        (4.86e+20, 3.197)
        (6.48e+20, 3.179)

    };
    \addlegendentry{Late-289M}

    \addplot[legend late_0_4b style] plot coordinates {
        (1.37e+20, 3.249)
        (2.74e+20, 3.177)
        (5.48e+20, 3.12)
        (8.22e+20, 3.09)
        (1.1e+21, 3.071)
    };
    \addlegendentry{Late-494M}

\addplot[legend late_0_9b style] plot coordinates {
        (2.78e+20, 3.134)
        (5.56e+20, 3.045)
        (1.11e+21, 2.976)
        (1.67e+21, 2.941)
        (2.22e+21, 2.92)

    };
    \addlegendentry{Late-1B}

\addplot[legend late_2_2b style] plot coordinates {
        (1.34e+21, 2.936)
        (2.69e+21, 2.859)
        (5.37e+21, 2.798)
        (8.06e+21, 2.765)

    };
    \addlegendentry{Late-2.4B}

\addplot[legend early_0_2b style] plot coordinates {
        (8.25e+19, 3.328)
        (1.65e+20, 3.259)
        (3.3e+20, 3.203)
        (4.95e+20, 3.175)
        (6.6e+20, 3.157)
    };
    \addlegendentry{Early-275M}

\addplot[legend early_0_4b style] plot coordinates {
        (1.39e+20, 3.232)
        (2.78e+20, 3.161)
        (5.57e+20, 3.103)
        (8.35e+20, 3.073)
        (1.11e+21, 3.054)
    };
    \addlegendentry{Early-464M}

    \addplot[legend early_0_9b style] plot coordinates {
        (2.8e+20, 3.119)
        (5.59e+20, 3.03)
        (1.12e+21, 2.961)
        (1.68e+21, 2.926)
        (2.24e+21, 2.904)
    };
    \addlegendentry{Early-932M}

\addplot[legend early_2_2b style] plot coordinates {
        (1.37e+21, 2.923)
        (2.74e+21, 2.846)
        (5.47e+21, 2.784)
        (8.21e+21, 2.752)
    };
    \addlegendentry{Early-2.28B}

\end{axis}
\end{tikzpicture}

    \end{subfigure}
    \begin{center}
        \ref{sharedlegend}
    \end{center}
    \caption{\textbf{早期融合与晚期融合:扩展训练FLOP数。} 我们比较了在扩展模型参数数量和训练令牌数量时,早期融合和晚期融合模型的表现。总体而言,早期融合在较小的模型规模下表现出轻微的优势,并且随着参数数量 $N$ 的增加,差距逐渐缩小。}
    \label{fig:early_vs_late_scaleflops}
    \vspace{3mm}
\end{figure*}
