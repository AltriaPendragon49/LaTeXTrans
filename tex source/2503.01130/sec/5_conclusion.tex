\vspace{-5pt}
\section{Conclusion}
\vspace{-5pt}
\label{sec:conclusion}

Room reidentification is a challenging yet crucial research area, with growing applications in fields like augmented reality and homecare robotics. In this paper, we introduce AirRoom, a training-free, object-aware approach for room reidentification. AirRoom leverages multi-level object-oriented features to capture both spatial and contextual information of indoor rooms. To evaluate AirRoom, we constructed four novel datasets specifically for room reidentification. Experimental results demonstrate its robustness to viewpoint variations and superior performance over state-of-the-art methods across nearly all metrics and datasets. Furthermore, the pipeline is highly flexible, maintaining high performance without relying on specific model configurations. Collectively, our work establishes AirRoom as a powerful and versatile solution for precise room reidentification, with broad potential for real-world applications.

\begin{center}
\textbf{Acknowledgments}  
\end{center}
% This work was partially supported by the DARPA grant DARPA-PS-23-13. The views and conclusions contained in this document are those of the authors and should not be interpreted as representing the official policies, either expressed or implied, of DARPA.
\begin{sloppypar}
\noindent This work was supported by the DARPA award HR00112490426. Any opinions, findings, conclusions, or recommendations expressed in this paper are those of the authors and do not necessarily reflect the views of DARPA.
\end{sloppypar}

