
\begin{figure}[t!]
    \centering
    \captionsetup{type=figure}
    \begin{subfigure}[t]{0.32\linewidth}
        \begin{tikzpicture}
    \begin{axis}[
        legend pos=north east,
        grid=both,
        grid style={line width=.1pt, draw=gray!10},
        major grid style={line width=.2pt,draw=gray!50},
        minor tick num=2,
        axis x line*=bottom,
        axis y line*=left,
        xtick={
         0.1,
        0.4,
        1
        },
        xticklabels={
         \texttt{100B},
        \texttt{400B},
        \texttt{1T},
        },
        xmin=0.05,
        xmax=1,
        width=1.3\linewidth,
        ylabel style={align=center, font=\scriptsize, yshift=-1ex},
        xlabel style={font=\scriptsize},
        title={\scriptsize{Image-Caption CE}},
        xlabel={\scriptsize{Tokens seen}},
        yticklabel style={font=\scriptsize},
        xticklabel style={font=\scriptsize},
        legend style={cells={align=left}, font=\scriptsize, fill opacity=0.7},
        legend cell align={left},
        mark options={solid},
    ]

\addplot[LateGradStart!75!LateGradEnd, thick, solid, mark=*, mark size=1.5pt] plot coordinates {
        (0.1, 2.329)
        (0.2, 2.274)
        (0.3, 2.255)
        (0.4, 2.234)
        (0.5, 2.224)
        (0.6, 2.212)
    };

\addplot[legend early_2_2b style] plot coordinates {
        (0.1, 2.447)
        (0.3, 2.34)
        (0.4, 2.321)
        (0.5, 2.297)
        (0.6, 2.284)
        (0.8, 2.274)
        (1,   2.256)
        (1.2, 2.251)
    };

\end{axis}
\end{tikzpicture}

    \end{subfigure}
    \begin{subfigure}[t]{0.32\linewidth}
        \begin{tikzpicture}
    \begin{axis}[
        legend pos=north east,
        grid=both,
        grid style={line width=.1pt, draw=gray!10},
        major grid style={line width=.2pt,draw=gray!50},
        minor tick num=2,
        axis x line*=bottom,
        axis y line*=left,
        xtick={
         0.1, 
        0.4,
        1
        },
        xticklabels={
         \texttt{100B}, 
        \texttt{400B},
        \texttt{1T},
        },
        xmin=0.05,
        xmax=1,
        width=1.3\linewidth,
        ylabel style={align=center, font=\scriptsize, yshift=-1ex},
        xlabel style={font=\scriptsize},
        title={\scriptsize{Interleaved CE}},
        xlabel={\scriptsize{Tokens seen}},
        yticklabel style={font=\scriptsize},
        xticklabel style={font=\scriptsize},
        legend style={cells={align=left}, font=\scriptsize, fill opacity=0.7}, %
        legend cell align={left},
        mark options={solid},
    ]





    



    \addplot[LateGradStart!75!LateGradEnd, thick, solid, mark=*, mark size=1.5pt] plot coordinates {
        (0.1, 2.582)
        (0.2, 2.558)        
        (0.3, 2.54)
        (0.4, 2.528)
        (0.5, 2.511)
        (0.6, 2.508)


    };


    \addplot[legend early_2_2b style] plot coordinates {
        (0.1, 2.601)
        (0.3, 2.556)
        (0.4, 2.544)
        (0.5, 2.534)       
        (0.6, 2.527) 
        (0.8, 2.513)
        (1,   2.506)
        (1.2, 2.501)
    };


    


    

    \end{axis}
\end{tikzpicture}








    







    


    \end{subfigure}
    \begin{subfigure}[t]{0.32\linewidth}
        \begin{tikzpicture}
    \begin{axis}[
        legend pos=north east,
        grid=both,
        grid style={line width=.1pt, draw=gray!10},
        major grid style={line width=.2pt,draw=gray!50},
        minor tick num=2,
        axis x line*=bottom,
        axis y line*=left,
        xtick={
         0.1,
        0.4,
        1
        },
        xticklabels={
        \texttt{100B},
        \texttt{400B},
        \texttt{1T},
        },
        xmin=0.05,
        xmax=1,
        width=1.3\linewidth,
        ylabel style={align=center, font=\scriptsize, yshift=-1ex},
        xlabel style={font=\scriptsize},
        title={\scriptsize{Text CE}},
        xlabel={\scriptsize{Tokens seen}},
        yticklabel style={font=\scriptsize},
        xticklabel style={font=\scriptsize},
        legend style={cells={align=left}, font=\scriptsize, fill opacity=0.7},
        legend cell align={left},
        mark options={solid},
    ]

\addplot[LateGradStart!75!LateGradEnd, thick, solid, mark=*, mark size=1.5pt] plot coordinates {
        (0.1, 2.791)
        (0.2, 2.779)
        (0.3, 2.765)
        (0.4, 2.756)
        (0.5, 2.748)
        (0.6, 2.741)
    };

    \addplot[legend early_2_2b style] plot coordinates {
        (0.1, 2.795)
        (0.3, 2.77)
        (0.4, 2.761)
        (0.5, 2.753)
        (0.6, 2.747)
        (0.8, 2.738)
        (1, 2.731)
        (1.2, 2.727)
    };

\end{axis}
\end{tikzpicture}

    \end{subfigure}

    \makebox[0.9\linewidth]{ %
        \begin{tikzpicture}
            \begin{axis}[
                hide axis, %
                xmin=0, xmax=0.5, ymin=0, ymax=1, %
                legend columns=2, %
                legend style={
                    at={(0.5, 1)}, %
                    anchor=north, %
                    /tikz/every even column/.append style={column sep=0.2cm}, %
                    scale=0.5, %
                    cells={align=left}, font=\footnotesize,
                },
            ]
            \addlegendimage{LateGradStart!75!LateGradEnd, thick, solid, mark=*, mark size=1.5pt}
            \addlegendentry{Late-init}
        
            
            \addlegendimage{legend early_2_2b style}
            \addlegendentry{Early-Init}
            \end{axis}
        \end{tikzpicture}
    }
    
    \vspace{-5cm}
    \caption{\textbf{Early vs late fusion when initializing the encoder and decoder.} Early-fusion can match the performance of late-fusion models when trained for longer. However, the gap is bigger on image-caption data.}
    \label{fig:early_vs_late_init_scaledata}
\end{figure}
