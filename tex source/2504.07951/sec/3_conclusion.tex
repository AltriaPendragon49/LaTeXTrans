

\section{\edit{Discussion and }Limitations} 

\cpar{\edit{Scaling laws for multimodal data mixtures.}} Our scaling laws study
spans different model configurations and training mixtures. While results
suggest that the scaling law coefficients remain largely consistent across
mixtures, a broader exploration of mixture variations is needed to validate this
observation and establish a unified scaling law that accounts for this factor.  

\cpar{\edit{Scaling laws and performance on downstream tasks.}} Similar to
previous scaling law studies, our analysis focuses on pretraining performance as
measured by the validation loss. However, the extent to which these findings
translate to downstream performance remains an open question and requires
further investigation.

\cpar{\edit{Extrapolation to larger scales.}} The accuracy of scaling law
predictions improves with increasing FLOPs~\cref{app:scaling_laws}.
\edit{Furthermore, we validate our laws when extrapolating to larger model sizes
(\cref{sec:scaling_laws_evaluation}).} However, whether these laws can be reliably
extrapolated to extremely large model sizes remains an open question. 

\cpar{\edit{High resolution and early-fusion models.}} Training early-fusion
models with high-resolution inputs leads to a significant increase in vision
tokens. While pooling techniques have been widely adopted for late-fusion
models, alternative approaches may be necessary for early fusion. \edit{Given
the similarity of early-fusion models to LLMs, it appears that techniques for
extending context length could be beneficial.}

\cpar{\edit{Scaling laws for multimodal MoEs models.}} For MoEs, we consider
only a single configuration (top-1 routing with 8 experts). \edit{We found this
configuration to work reasonably well in our setup, and follow a standard MoEs
implementation}. However, the findings may vary when optimizing \edit{more} the
MoE architecture or exploring different load-balancing, routing strategies
\edit{or different experts implementations}.

\section{Conclusion} 
We explore various strategies for compute-optimal pretraining of native
multimodal models. We found the NMMs follow similar scaling laws to those of
LLMs. Contrary to common belief, we find no inherent advantage in adopting
late-fusion architectures over early-fusion ones. While both architectures
exhibit similar scaling properties, early-fusion models are more efficient to
train and outperform late-fusion models at lower compute budgets. Furthermore,
we show that sparse architectures encourage modality-specific specialization,
leading to performance improvements while maintaining the same inference cost.




\section*{\edit{Acknowledgment}} We thank Philipp Dufter, Samira Abnar, Xiujun
Li, Zhe Gan, Alexander Toshev, Yinfei Yang, Dan Busbridge, and Jason Ramapuram
for many fruitful discussions. We thank Denise Hui, and Samy Bengio for infra
and compute support. Finally, we thank, Louis Béthune, Pierre Ablin, Marco
Cuturi, and the MLR team at Apple for their support throughout the project.