\begin{figure}[htbp]
        \centering
            \vspace{-16pt}
        \foreach \row in {0,1,2,3}{
            \foreach \col in {ipr_mse_zoomed_in, timeshot}{
                \includegraphics[height=50pt]{rebuttal-figures/time/seed=\row/\col.pdf}
                \hspace{10pt}
            }
            \\ 
            \foreach \col in {ipr_mse_zoomed_in, timeshot}{
                \label{fig:\row\col}
            }
            \vspace{4pt} 
        }
        \vspace{4pt}
        \hrule
        \vspace{4pt}
        \foreach \col in {ipr_mse, timeshot}{
            \includegraphics[height=60pt]{rebuttal-figures/time/seed=0_gaussian/\col.pdf}
            \hspace{10pt}
        }
        \\ 
        \foreach \col in {ipr_mse, timeshot}{
            \label{fig:gaussian_\col}
        }
        \caption{
(\textbf{Top}) Four initializations trained on $\texttt{NLGP}(g=100)$ with $\xi_0 = 0.3$ and $\xi_1 = 0.7$.
As expected, weights always localize.
In (Left, First) we plot IPR for empirical and analytical receptive fields (RFs) across time (defined as (\# of gradient steps) $\times \ \, \tau$, the learning rate).
In (Left, Second) we plot the time-evolution of $\ell_2$ distance between the empirical and analytical RFs.
In (Left, Third) we zoom in on (Left, First), restricting the range to $[0,0.1]$ to more closely see divergence in IPR early in training.
In (Right, First) and (Right, Second), we snapshot the empirical and analytical RFs at a time \emph{before} and \emph{just after}, respectively, the analytical model breaks down  (according to IPR and $\ell_2$ distance) due to localization.
Finally, in (Right, Third), we snapshot \emph{at the end} of the training period.
(\textbf{Bottom}) Same initialization as first row in \textbf{top}, but trained on $\texttt{NLGP}(g=0.01)$ data, again with $\xi_0 = 0.3$ and $\xi_1 = 0.7$.
As expected, weights do not localize.
We plot the same quantities as above, but here the predictions of our analytical model hold \emph{throughout} the entire training process as localization never emerges and so assumption (A3) is not violated as above.
\label{fig:time}
}
    \end{figure}    
