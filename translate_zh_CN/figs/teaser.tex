
\begin{figure}[t!]
    \centering
    \captionsetup{type=figure}
    \begin{subfigure}[t]{0.48\linewidth}
        \begin{tikzpicture}[
    spy using outlines={rectangle, magnification=2, size=0.5in, connect spies}
]
    \begin{axis}[
        legend pos=south west,
        grid=both,
        grid style={line width=.1pt, draw=gray!10},
        major grid style={line width=.2pt,draw=gray!50},
        minor tick num=2,
        axis x line*=bottom,
        axis y line*=left,
        xmode=log,
        log basis x=10,
        height=2.5in,
        width=1.1\linewidth,
        xtick distance=1e02,
        ylabel={\footnotesize{Validation Loss ($L$)}},
        ytick distance=0.5,
        yticklabel style={font=\footnotesize, /pgf/number format/fixed, /pgf/number format/precision=2},
        xlabel={\footnotesize{FLOPs ($C$)}},
        xticklabel style={font=\footnotesize},
        legend style={cells={align=left}, font=\footnotesize, fill opacity=0.7},
        mark options={solid},
    ]

\addplot[line width=1.5pt, dashdotdotted, color=EarlyGradStart!50!EarlyGradEnd, samples=50, domain=5e18:5e23] {29.57440597247478*x^-0.04919003913934263};
    \addlegendentry{\scalebox{0.9}{Early: $L \propto C^{-0.0492}$}};
    \addplot[line width=1.5pt, dashdotdotted, color=LateGradStart!50!LateGradEnd, samples=50, domain=5e18:5e23] {30.038509408784137*x^-0.049424801983112776};
    \addlegendentry{\scalebox{0.9}{Late: $L \propto C^{-0.0494}$}};

    \addplot[line width=1.5pt, dashdotdotted, color=MOEGradStart!50!MOEGradEnd, samples=50, domain=5e18:5e23] {26.287135104499598*x^-0.04742807363789748};
    \addlegendentry{\scalebox{0.9}{MoE: $L \propto C^{-0.0474}$}};

    \begin{scope}
        \spy on (1.5,3.2) in node [right] at (3.8,3.5);
    \end{scope}

    \end{axis}
\end{tikzpicture}

    \end{subfigure}
    \hfill
    \begin{subfigure}[t]{0.48\linewidth}
        \begin{tikzpicture}
    \begin{axis}[
        legend pos=north west,
        grid=both,
        grid style={line width=.1pt, draw=gray!10},
        major grid style={line width=.2pt,draw=gray!50},
        minor tick num=2,
        axis x line*=bottom,
        axis y line*=left,
        xmode=log, %
        log basis x=10, %
        height=2.5in,
        width=1.1\linewidth,
        xtick distance=1e02,
        ylabel={\footnotesize{$N/D$}},
        ylabel style={yshift=-1ex},
        xlabel={\footnotesize{FLOPs ($C$)}},
        yticklabel style={font=\footnotesize},
        xticklabel style={font=\footnotesize},
        legend style={cells={align=left}, font=\footnotesize, fill opacity=0.7}, %
        mark options={solid},
    ]

    \addplot[line width=1.7pt, dashdotdotted, color=EarlyGradStart!50!EarlyGradEnd, samples=50, domain=5e18:5e23] {((780.3128750575453^-1)*x^(0.05302278709608001)};
    \addlegendentry{\scalebox{0.9}{Early: $\frac{N}{D} \propto C^{0.053}$}};
    \addplot[line width=1.7pt, dashdotdotted, color=LateGradStart!50!LateGradEnd, samples=50, domain=5e18:5e23] {(1880.7495777617235^-1)*x^(0.07616482680724829)};
    \addlegendentry{\scalebox{0.9}{Late: $\frac{N}{D} \propto C^{0.076}$}};

    \addplot[line width=1.7pt, dashdotdotted, color=MOEGradStart!50!MOEGradEnd, samples=50, domain=5e18:5e23] {((5.024013439838208e-05)^-1)*x^(-0.31205022008476146)};
    \addlegendentry{\scalebox{0.9}{MoE: $\frac{N}{D} \propto C^{-0.312}$}};

    \end{axis}
\end{tikzpicture}
    \end{subfigure}
    \vspace{-3mm}
    \caption{\textbf{原生多模态模型的缩放属性。}基于在\cref{sec:scaling_laws_early}中对缩放定律的研究,我们观察到:(1) 当使用相同的计算预算$C$(以FLOP为单位)进行训练时,早期融合模型和晚期融合模型提供相当的验证损失$L$;(2) 这种性能是通过在参数$N$和训练词元数量$D$之间进行不同的权衡来实现的,其中早期融合模型需要的参数更少。
\edit{;(3) 对于给定的FLOP预算,稀疏早期融合模型实现了更低的损失并且需要更多的训练词元。} 
    }
    \label{fig:teaser}
\end{figure}
