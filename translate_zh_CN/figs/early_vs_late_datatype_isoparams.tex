
\begin{figure*}[t!]
    \centering
    \captionsetup{type=figure}
    \begin{subfigure}[t]{0.32\linewidth}
         \begin{tikzpicture}
    \begin{axis}[
        legend pos=north east,
        grid=both,
        grid style={line width=.1pt, draw=gray!10},
        major grid style={line width=.2pt,draw=gray!50},
        minor tick num=2,
        axis x line*=bottom,
        axis y line*=left,
        xtick={
            0,
            18,
            27,
            45,
            63,
            72
        },
        xticklabels={
            0,
            18,
            27,
            45,
            63,
            72
        },
        xmin=0,
        xmax=72,
        width=1.1\linewidth,
        ylabel style={align=center, font=\tiny, yshift=-1ex},
        xlabel style={font=\footnotesize},
        title={\tiny{Paired CE}},
        title style={yshift=-1ex},
        xlabel={\tiny{\% of Interleaved}},
        yticklabel style={font=\tiny},
        xticklabel style={font=\tiny},
        legend style={cells={align=left}, font=\tiny},
        legend cell align={left},
        mark options={solid},
    ]

\addplot[legend late style] plot coordinates {
    (0, 2.287)
    (18, 2.324)
    (27, 2.346)
    (45, 2.413)
    (63, 2.53)
    (72, 2.626)
};

\addplot[EarlyGradStart!5!EarlyGradEnd, thick, solid, mark=*, mark size=1.75pt]  plot coordinates {
    (0, 2.292)
    (18, 2.338)
    (27, 2.357)
    (45, 2.431)
    (63, 2.527)
    (72, 2.607)
};

\addplot[legend early style, mark size=1.75pt] plot coordinates {
    (0, 2.277)
    (18, 2.313)
    (27, 2.34)
    (45, 2.41)
    (63, 2.514)
    (72, 2.586)
};

\addplot[EarlyGradStart!95!EarlyGradEnd, thick, solid, mark=*, mark size=1.75pt] plot coordinates {
    (0, 2.263)
    (18, 2.304)
    (27, 2.323)
    (45, 2.383)
    (63, 2.494)
    (72, 2.565)
};

\end{axis}
\end{tikzpicture}

    \end{subfigure}
    \begin{subfigure}[t]{0.32\linewidth}
        


\begin{tikzpicture}
    \begin{axis}[
        legend pos=north east,
        grid=both,
        grid style={line width=.1pt, draw=gray!10},
        major grid style={line width=.2pt,draw=gray!50},
        minor tick num=2,
        axis x line*=bottom,
        axis y line*=left,
        xtick={
            18, 
            27, 
            45, 
            63, 
            72
        },
        xticklabels={
            18, 
            27, 
            45, 
            63, 
            72
        },
        xmin=18,
        xmax=90,
        width=1.1\linewidth,
        ylabel style={align=center, font=\tiny, yshift=-1ex},
        xlabel style={font=\footnotesize},
        title={\tiny{Interleaved CE}},
        title style={yshift=-1ex}, %
        xlabel={\tiny{\% of Interleaved}},
        yticklabel style={font=\tiny},
        xticklabel style={font=\tiny},
        legend style={cells={align=left}, font=\tiny}, %
        legend cell align={left},
        mark options={solid},
    ]



    

    

    
    


\addplot[legend late style] plot coordinates {
    (18, 2.773) 
    (27, 2.726) 
    (45, 2.666) 
    (63, 2.63) 
    (72, 2.615) 
    (90, 2.594)
};

\addplot[EarlyGradStart!5!EarlyGradEnd, thick, solid, mark=*, mark size=1.75pt] plot coordinates {
    (18, 2.781) 
    (27, 2.735) 
    (45, 2.673) 
    (63, 2.62) 
    (72, 2.605) 
    (90, 2.582)
};

\addplot[legend early style, mark size=1.75pt] plot coordinates {
    (18, 2.767) 
    (27, 2.719) 
    (45, 2.656) 
    (63, 2.618) 
    (72, 2.601) 
    (90, 2.578)
};

\addplot[EarlyGradStart!95!EarlyGradEnd, thick, solid, mark=*, mark size=1.75pt] plot coordinates {
    (18, 2.752) 
    (27, 2.695) 
    (45, 2.639) 
    (63, 2.586) 
    (72, 2.584)
    (90, 2.561)
};


    \end{axis}
\end{tikzpicture}
    \end{subfigure}
    \begin{subfigure}[t]{0.32\linewidth}
         \begin{tikzpicture}
    \begin{axis}[
        legend pos=north east,
        grid=both,
        grid style={line width=.1pt, draw=gray!10},
        major grid style={line width=.2pt,draw=gray!50},
        minor tick num=2,
        axis x line*=bottom,
        axis y line*=left,
        xtick={
            0,
            18,
            27,
            45,
            63,
            72,
        90
        },
        xticklabels={
            0,
            18,
            27,
            45,
            63,
            72,
            90
        },
        xmin=0,
        xmax=90,
        width=1.1\linewidth,
        ylabel style={align=center, font=\tiny, yshift=-1ex},
        xlabel style={font=\tiny},
        title={\tiny{Text CE}},
        title style={yshift=-1ex},
        xlabel={\tiny{\% of Interleaved}},
        yticklabel style={font=\tiny},
        xticklabel style={font=\tiny},
        legend style={cells={align=left}, font=\tiny},
        legend cell align={left},
        mark options={solid},
    ]

\addplot[legend late style] plot coordinates {
(0, 3.033)
(18, 2.941)
(27, 2.918)
(45, 2.887)
(63, 2.867)
(72, 2.859)
(90, 2.85)
};

\addplot[EarlyGradStart!5!EarlyGradEnd, thick, solid, mark=*, mark size=1.75pt] plot coordinates {
(0, 3.044)
(18, 2.952)
(27, 2.926)
(45, 2.891)
(63, 2.871)
(72, 2.863)
(90, 2.853)
};

\addplot[legend early style, mark size=1.75pt] plot coordinates {
(0, 3.046)
(18, 2.953)
(27, 2.927)
(45, 2.894)
(63, 2.873)
(72, 2.864)
(90, 2.853)
};

\addplot[EarlyGradStart!95!EarlyGradEnd, thick, solid, mark=*, mark size=1.75pt] plot coordinates {
(0, 3.017)
(18, 2.92)
(27, 2.894)
(45, 2.858)
(63, 2.838)
(72, 2.83)
(90, 2.836)
};

    \end{axis}
\end{tikzpicture}

    \end{subfigure}
    
    \makebox[0.9\linewidth]{ %
        \begin{tikzpicture}
            \begin{axis}[
                hide axis, %
                xmin=0, xmax=0.5, ymin=0, ymax=1, %
                legend columns=4, %
                legend style={
                    at={(0.5, 1)}, %
                    anchor=north, %
                    /tikz/every even column/.append style={column sep=0.2cm}, %
                    scale=0.5, %
                    cells={align=left}, font=\footnotesize,
                },
            ]

                \addlegendimage{legend late style}
                \addlegendentry{L}
            
                
               \addlegendimage{EarlyGradStart!10!EarlyGradEnd, thick, solid, mark=*, mark size=1.75pt}
                \addlegendentry{E (Text)}
            
                
                \addlegendimage{legend early style, mark size=1.75pt}
                \addlegendentry{E (FLOPs)}
            
                
               \addlegendimage{EarlyGradStart!95!EarlyGradEnd, thick, solid, mark=*, mark size=1.75pt}
                \addlegendentry{E (Params)}
            \end{axis}
        \end{tikzpicture}
    }
    
    \vspace{-5cm}
    
    \caption{\textbf{早期融合与晚期融合:改变训练混合物和早期融合配置。} 我们改变训练混合物并绘制不同早期融合模型配置的最终训练损失。对于相同的总参数数量,早期融合始终优于晚期融合。}
    \label{fig:early_vs_late_datatype_isoparams}
\end{figure*}
